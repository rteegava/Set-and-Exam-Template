\documentclass[conference]{styles/acmsiggraph}

\usepackage{comment} % enables the use of multi-line comments (\ifx \fi)
\usepackage{lipsum} %This package just generates Lorem Ipsum filler text.
%\usepackage{fullpage} % changes the margin
\usepackage{enumitem} % for customizing enumerate tags
\usepackage{amsmath,amsthm,thmtools, amssymb}
\usepackage{listings}
\usepackage{graphicx}
\usepackage{etoolbox}   % for booleans and much more
\usepackage{verbatim}   % for the comment environment
\usepackage[dvipsnames]{xcolor}
\usepackage{fancyvrb}
\usepackage{hyperref}
\usepackage{menukeys}
\usepackage{titlesec}
\usepackage{pifont}
\setlength{\parskip}{.8mm}

\let\oldproofname=\proofname
\renewcommand{\proofname}{\rm\bf{\oldproofname}}

\declaretheorem[numbered = no, shaded={rulecolor=Gray,rulewidth=1pt, bgcolor={rgb}{0.9,0.9,0.9}}]{Theorem} %gray
\declaretheorem[numbered = no, shaded={rulecolor=Red,rulewidth=1pt, bgcolor={rgb}{1,0.8,0.8}}]{Lemma}
\declaretheorem[numbered = no, shaded={rulecolor=Green,rulewidth=1pt, bgcolor={rgb}{0.8,1,0.8}}]{Proposition}
\declaretheorem[numbered = no, shaded={rulecolor=Blue,rulewidth=1pt, bgcolor={rgb}{0.8,0.8,1}}]{Corollary}
\renewcommand{\qedsymbol}{\ensuremath{\includegraphics[width=0.25in, height=0.25in]{OIP.jpg}}}
%if you want a legitimate QED symbol, here is \null\hfill\textcolor{green}{\ding{52}}

\title{\huge Problem Set [Insert Number] \\ \LARGE {[Insert Class]}}
\author{\Large Name \\ \textcolor{blue}{[insert]@caltech.edu}}
\pdfauthor{Name}


\hypersetup{
	colorlinks=true,
	linkcolor=magenta,
	filecolor=magenta,
	urlcolor=blue,
}

% redefine \VerbatimInput
\RecustomVerbatimCommand{\VerbatimInput}{VerbatimInput}%
{fontsize=\footnotesize,
 %
 frame=lines,  % top and bottom rule only
 framesep=2em, % separation between frame and text
 rulecolor=\color{Gray},
 %
 label=\fbox{\color{Black}\textbf{OUTPUT}},
 labelposition=topline,
 %
 commandchars=\|\(\), % escape character and argument delimiters for
                      % commands within the verbatim
 commentchar=*        % comment character
}

% convenient norm symbol
\renewcommand{\vec}[1]{\mathbf{#1}}

\titlespacing*{\section}{0pt}{5.5ex plus 1ex minus .2ex}{2ex}
\titlespacing*{\subsection}{0pt}{3ex}{2ex}

\setcounter{secnumdepth}{4}	
\renewcommand\theparagraph{\thesubsubsection.\arabic{paragraph}}	
\newcommand\subsubsubsection{\paragraph}

\setlength{\parskip}{0.5em}

% a macro for hiding answers
\newbool{hideanswers}
\setbool{hideanswers}{false}
\newenvironment{answer}{}{}
\ifbool{hideanswers}{\AtBeginEnvironment{answer}{\comment} %
\AtEndEnvironment{answer}{\endcomment}}{}

\newcommand{\points}[1]{\hfill \normalfont{(\textit{#1 points})}}
\newcommand{\pointsin}[1]{\normalfont{(\textit{#1 points})}}

\begin{document}
\maketitle


%%%%%%%%%%%%%%%%%%
%   Question #1  %
%%%%%%%%%%%%%%%%%%
\section*{Question 1 \points{[number of points]}}
[Insert Problem Statement]
%%%%%%%%%%%%%%%%%%
%   Answer #1 
%%%%%%%%%%%%%%%%%%

\begin{answer}
\rule{\textwidth}{0.4pt}
	
	\textbf{Discussion:} Insert any prefacing needed for the question, such as the approach for the proof or any preliminary work.
	
	\begin{proof} 
	Write your actual proof here. The following boxes can be used to delineate theorems, lemmas, propositions, and corollaries respectively. Just replace the filler text with your actual verbiage. 
	\begin{Theorem}
	$\text{amogus}$
	\end{Theorem}
	%%%%%%%%%%%%%%%%%%%%
	\begin{Lemma}
	$\text{amogus}$
	\end{Lemma}
    %%%%%%%%%%%%%%%%%%%%	
	\begin{Proposition}
	$\text{amogus}$
	\end{Proposition}
    %%%%%%%%%%%%%%%%%%%%
	\begin{Corollary}
	$\text{amogus}$
	\end{Corollary}
	%%%%%%%%%%%%%%%%%%%%
	Here are some shorthand "word" commands I made for some common math operations.
	$$\cis \qquad \lcm \qquad \argmin \qquad \argmax \qquad \coker \qquad \Coker \qquad \dom \qquad \End \qquad \ext$$
	$$\FixedPt \qquad \Gal \qquad \GL \qquad \Hom \qquad \Id \qquad \im \qquad \Ker \qquad \Mor \qquad \pred \qquad \pr$$
	$$\rank \qquad \ran \qquad \Spec \qquad \Tr \qquad \Var \qquad \e{x} \qquad \p{x}$$
	Here are some shorthand "function" commands I made.
	$$\bij \qquad \inj \qquad \surj \qquad \bicond \qquad \cond \qquad \restrict{a,b} \qquad \from$$
	Here are some geometry and calculus commands I made.
	$$\dang \qquad \ray{AB} \qquad \seg{AB} \qquad x\dg \qquad \dydx \qquad \ddx$$
	Here are some shorthand "matrix" commands I made.
	$$\bm{1 & 2 & 3 \\ 4 & 5 & 6 \\ 7 & 8 & 9} \qquad \vm{1 & 2 & 3 \\ 4 & 5 & 6 \\ 7 & 8 & 9}$$
	Here are some common set commands I made. 
	$$\CC \qquad \EE \qquad \FF \qquad \NN \qquad \PP \qquad \QQ \qquad \RR \qquad \ZZ$$
	Here are some summation and product commands I made.
	$$\cycsum \qquad \symsum \qquad \cycprod \qquad \symprod$$
	Here are some operational commands I made.
	$$\ol{amogus} \qquad \ul{amogus} \qquad \ub{amogus} \qquad \ts{amogus} \qquad \mf{amogus} \qquad \tt{amogus}$$
	$$\abs{x} \qquad \norm{x} \qquad \floor{x} \qquad \ceil{x} \qquad \del{x} \qquad \cbr{x} \qquad \sbr{x}$$
	\qedhere
    \end{proof}
\rule{\textwidth}{0.4pt}
\end{answer}


\end{document}
